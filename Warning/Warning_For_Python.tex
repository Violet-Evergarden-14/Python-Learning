\documentclass{article}

\usepackage{ctex, tabularray, fancyvrb, beramono, booktabs, array, geometry, float}


\title{关于Python的注意事项}
\author{X}

\makeatletter % 允许修改带@的内部命令
\renewcommand{\verbatim@font}{\normalfont} % 将verbatim字体设为正文默认字体
\makeatother % 恢复默认


\begin{document}
\maketitle

\section{转义字符}
  \begin{table}[H]
    \centering
    \caption{Python转义字符示例}
    \begin{tblr}
    {
      vline{2} = {-}{},
      hline{1,5} = {-}{},
      hline{2} = {-}{0.05em},
    }
      符号                               & 说明                  \\
      \textbackslash{}\textbackslash{} & 反斜杠\textbackslash{} \\
      \textbackslash{}'                & 单引号'                \\
      \textbackslash{}"                & 双引号"                
    \end{tblr}
  \end{table}

\section{原始字符串}
  利用r``text"可以使后文以文本直接显示,不进行转义操作

\section{print操作内部换行}
  使用\verb|\n|再加上\verb|\|换行,如果你嫌同一行太长,也可以使用\verb|\|将同一行的字隔断,隔断后,下一行打印时自动接在上一行

\section{长字符}
  利用\verb|```text'''或"""text"""|显示长字符串,长字符串段内换行有效且不会报错,转义会正常进行

\section{算符顺序}
  \begin{table}[H]
    \centering
    \caption{运算符优先级从低(1)到高(17)}
    \begin{tabular}{p{1.3cm}p{3cm}p{8.2cm}}
      \toprule
      优先级 & 运算符 & 描述 \\
      \midrule
      1 & lambda & Lambda表达式 \\
      2 & if - else & 条件表达式 \\
      3 & or & 布尔“或” \\
      4 & and & 布尔“与” \\
      5 & not x & 布尔“非” \\
      6 & in, not in, is, is not, $<$, $\leq$, $>$, $\geq$, $\neq$, $=$ & 成员测试,同一性测试,比较 \\
      7 & | & 按位或 \\
      8 & $\wedge$ & 按位异或 \\
      9 & $\&$ & 按位与 \\
      10 & $\ll$, $\gg$ & 移位 \\
      11 & +, - & 加法,减法 \\
      12 & *, @, /, //, $\%$ & 乘法,矩阵乘法,除法,地板除,取余数 \\
      13 & +x, -x, $\sim$x & 正号,负号,按位翻转 \\
      14 & ** & 指数 \\
      15 & await x & Await表达式 \\
      16 & x[index], x[index:index], x(arguments...), x.attribute & 下标,切片,函数调用,属性引用 \\
      17 & (expressions...), [expressions...], \{key:value...\}, \{expressions...\} & 绑定或元组显示,列表显示,字典显示,集合显示 \\
      \bottomrule
    \end{tabular}
  \end{table}

\section{符号}
  \verb|<=|号中\verb|<号和=号|必须挨在一起

\section{while循环语句}
  break操作\footnote{break操作会结束离它最近的循环}可以直接结束while循环,或者while循环未达条件时也会自动退出

\section{浮点数}
  \subsection{浮点数的精确计算}
    利用decimal模块可以使浮点数精确,操作如下:
    \begin{verbatim}
      import decimal
      x = decimal.Decimal(`欲输入的数字')
    \end{verbatim}\par
    之后以此类推,将所有浮点数这样表示即可进行精确运算\par
  \subsection{浮点数的科学表示}
    使用aEb表示$a \cdot 10^b$
  
\section{虚数}
  用x = a + bj来表示,以浮点数形式储存,x.real表示实部,x.imag表示虚部

\section{运算}
  \begin{table}[H]
    \centering
    \caption{python运算操作示例}
    \begin{tblr}
      {
        vline{2} = {1-10}{},
        hline{1,11} = {-}{0.08em},
        hline{2} = {-}{0.05em},
      }
      操作              & 结果                   \\
      x \% y          & 取余                   \\
      x // y          & 向下取整                 \\
      abs(x)          & 绝对值或复数的模                  \\
      int(x)          & 将x转换为整数              \\
      float(x)        & 将x转换为浮点数             \\
      complex(re, im) & 返回一个复数,re是其实部,im是其虚部 \\
      c.conjugate()   & 返回c的共轭复数             \\
      divmod(x, y)    & 返回x除以y的整除结果以及余数      \\
      pow(x, y)或x** y & x的y次方                 
    \end{tblr}
  \end{table}\par
  注意:以上类型可以对字符串进行操作,但是字符串表示一个复数时,中间不能有空格,否则会报错

\section{bool类型}
  \subsection{输出为False的内容}
    False\footnote{False类型输出值可能为下文所述的False,None,各种0,空阔号(),空字符串、序列或集合,但值都为0},None,值等于0的数字,空序列及集合
  \subsection{not逻辑运算符}
    not逻辑运算符会输出与not后内容相反的bool类型的值

\section{短路逻辑,针对and和or算符}
  \subsection{and}
    and运算符打印最前面的能决定bool类型Ture或False的变量\par
    例:bool类型为True,打印最后一个变量;bool类型为False,打印False
  \subsection{or}
    or运算符打印第一个非False类型的变量,若全为False则打印False

\section{分支结构(三元运算符)}
  形如以下的分支:
  \begin{verbatim}
    变量 = 值1 if 条件 else 值2
  \end{verbatim}\par
  即为三元运算符。\par 注意值1与值2可以是任意可行的变量类型,可以理解为=号后是一个整体,先进行这个整体的判断,得到结果(值1或值2),再将结果赋值给变量。

\section{input语句注意事项}
  input(``text")语句输入的内容默认是字符串,要进行运算,必须先转换为int,float等数字类型

\section{换行}
  \subsection{默认换行}
    输出换行符为\verb|(\n)|,print函数会在输出末尾自动添加\verb|(\n)|,导致后续输出换行,另外print()操作可以理解为不输入任何内容的换行操作
  \subsection{修改换行命令}
    通过\verb|end|参数修改结尾字符(默认值为\verb|``\n"|),实现自定义分隔或不换行,比如:
    \begin{verbatim}
      print(`text', end = `')
    \end{verbatim}
    实现不换行的命令;或者
    \begin{verbatim}
      print(`text', end = `1')
    \end{verbatim}
    使得在每一个输出后添一个1,但不进行换行操作;又或者
    \begin{verbatim}
      print(`text', end = `1 \n')
    \end{verbatim}
    使得在每一个输出后添一个1,并进行换行操作

\section{range生成序列}
  注意range是左包含右不包含的,且输出为int类型,有以下三种类型:
  \begin{table}[H]
    \centering
    \caption{range操作}
    \begin{tblr}{
        vline{2} = {-}{},
        hline{1,5} = {-}{0.08em},
        hline{2} = {-}{0.05em},
      }
      写法           & 意义            \\
      range(a)     & {[}0,a)间隔1取一个 \\
      range(a,b)   & {[}a,b)间隔1取一个 \\
      range(a,b,x) & {[}a,b)间隔x取一个 
    \end{tblr}
  \end{table}\par
  需要注意的是:a,b,x必须为int类型

\section{循环中的else}
  值得注意的是,这里面的else是与for或while并列,只有在循环正常结束(没有break打断)时执行(从某种意义上讲,else中的语句在循环寿终正寝,不被打断才执行)
  \subsection{for-else循环}
    语法结构及意义如下:
    \begin{verbatim}
      for 变量 in 可迭代对象:    # 循环体(处理每个元素)
        if 条件:
          statement
          break    # 打断循环(此时else不执行)
      else:
        statement    # 循环正常结束(无break)时执行
    \end{verbatim}
  \subsection{while-else循环}
    语法结构及意义如下:
    \begin{verbatim}
      while 条件:    # 循环体
        if 条件:
          statement
          break    # 打断循环(此时else不执行)
      else:
        statement    # 循环条件不成立(无break)时执行
    \end{verbatim}

\section{列表}
  \subsection{列表的构建以及元素的获取}
    \subsubsection{单个元素}
      我们构建如下列表:
      \begin{verbatim}
        list = [element_0, element_1, element_2, ……,element_n-1]
      \end{verbatim}\par
      当提取该序列(列表)的某个元素时,我们用
      \begin{verbatim}
        list[x]
      \end{verbatim}\par
      来提取序列的第x+1个元素\footnote{类似的,所有可迭代序列都可以用这种方法得到序列中特定的元素}(即list中的\verb|element_x|),需要注意的是,list中的序列从\verb|element_0|开始,而不是\verb|element_1|
    \subsubsection{元素位置的获取}
      我们用
      \begin{verbatim}
        list.index(element, start, end)
      \end{verbatim}\par
      来获得从start到end的第一个element元素的下标索引
    \subsubsection{切片}
      我们可以用
      \begin{verbatim}
        list[x: y: a]
      \end{verbatim}\par
      来获取list中从第x+1个元素(包含)到第y+1个元素(不包含)每a个元素取一个所构成的子序列(又名``切片")\par
      需要注意的是:\par
      1.x的默认值为0,y的默认值为序列的元素数,a的默认值为1,最后的:a可以不写\par
      2.若a, x, y同为负数就按照相应的规则倒序获取即可
    \subsubsection{列表修改1——增}
      我们用
      \begin{verbatim}
        list.append(element)
      \end{verbatim}\par
      在列表末尾添加一个元素\footnote{注意这里仅能添加一个元素};或者用
      \begin{verbatim}
        list.extend(iterable)
      \end{verbatim}\par
      在列表末尾添加一组可迭代对象\footnote{iterable的意思是可迭代对象,例如数组,列表,字符串等};也可以用
      \begin{verbatim}
        list.insert(location, element)
      \end{verbatim}\par
      来对某个位置插入一个元素
    \subsubsection{列表修改2——删}
      我们用
      \begin{verbatim}
        list.remove(element)
      \end{verbatim}\par
      来删除某个元素(element),注意,这里只会删除列表中的第一个element元素;或者用
      \begin{verbatim}
        list.pop(number)
      \end{verbatim}\par
      来删除序号为number的元素;也可以用
      \begin{verbatim}
        list.clear()
      \end{verbatim}\par
      来直接清空list列表
    \subsubsection{列表修改3——改}
      我们用
      \begin{verbatim}
        list[start: end: step] = iterable
      \end{verbatim}\par
      修改列表元素,注意这里的替换包含了start,不包含end,当$start = end$时,相当于在start位置前添加可迭代对象(也可以理解为对list的切片进行赋值),或者用
      \begin{verbatim}
        list[number] = element
      \end{verbatim}\par
      来将list中下标为number的元素更改为元素element,结合上方index指令的内容,我们可以用
      \begin{verbatim}
        list[list.index(element)] = element*
      \end{verbatim}\par
      将第一个element元素换成element*元素

  \subsection{列表的``运算''}
    \subsubsection{``加法''}
      ``加法''操作:
      \begin{verbatim}
        list1 + list2
      \end{verbatim}\par
      将list2的序列接到list1后,形成新列表

    \subsubsection{``乘法''}
      ``乘法'操作':
      \begin{verbatim}
        list * number
      \end{verbatim}\par
      是将list中元素重复number次,形成新列表
  
    \subsubsection{拷贝}
      我们用(浅拷贝)
      \begin{verbatim}
        list1 = list.copy()
      \end{verbatim}\par
      或者
      \begin{verbatim}
        list1 = list[:]
      \end{verbatim}\par
      又或者
      \begin{verbatim}
        import copy

        list = [……]
        list1 = copy.copy(list)
      \end{verbatim}
      来实现对整个列表外围对象的拷贝(外围内存位置改变,但是内层如果有嵌套对象,内层列表的对象会指向相同的内存位置)\footnote{使用$list1 = list$只实现了对内存位置的拷贝}\par
      如果要实现对列表完全的拷贝(修改原对象不会对之后的列表造成影响),我们可以用copy模块的deepcopy函数,操作如下:
      \begin{verbatim}
        import copy

        matrix = [……]
        matrix1 = copy.deepcopy(matrix)
      \end{verbatim}

    \subsubsection{列表推导式}
      基本语法为:
      \begin{verbatim}
        list = [expression for target in  if condition]
      \end{verbatim}\par
      将其转换为for循环语句为:
      \begin{verbatim}
        list = []
        for target in :
          if condition:
            list.append(expression)
      \end{verbatim}
      当然,如果是直接将原列表改为列表推导式的列表,那就需要将原列表再赋值为list;我们可以表示嵌套的列表推导式如下:
      \begin{verbatim}
        list = [expression for target1 in iterable1 if condition1
                           for target2 in iterable2 if condition2
                           ...
                           for targetN in iterableN if conditionN]
      \end{verbatim}\par
      用for循环语句来解释就是:
      \begin{verbatim}
        list = []
        for target1 in iterable1:
          if condition1:
            for target2 in iterable2:
              if condition2:
                ...
                  for targetN in iterableN:
                    if conditionN:
                      list.append(expression)
      \end{verbatim}

  \subsection{嵌套列表}
    \subsubsection{嵌套列表的构建}
      以如下方式构建一个嵌套列表:
      \begin{verbatim}
        matrix = [[0, 0, 0], [0, 0, 0], [0, 0, 0]]
      \end{verbatim}\par
      当然,也可以写作:
      \begin{verbatim}
        matrix = [[0, 0, 0],
                  [0, 0, 0],
                  [0, 0, 0]]
      \end{verbatim}\par
      或者,我们用
      \begin{verbatim}
        A = [0] * 3 

        for i in range(3):
          A[i] = [0, 0, 0]
      \end{verbatim}\par
      来构建一个初始列表
    \subsubsection{嵌套列表的读取}
      我们用:
      \begin{verbatim}
        for i in matrix:
          for each in i:
            print(each)
      \end{verbatim}\par
      来读取列表中的所有元素,或者用
      \begin{verbatim}
        matrix[i]
      \end{verbatim}\par
      来读取第i+1行的列表,或者用
      \begin{verbatim}
        matrix[i][j]
      \end{verbatim}\par
      来读取第i+1行第j+1列的元素

  \subsection{其他注意}
    \subsubsection{列表长度的获取}
      利用\verb|len(list)|函数可以得到列表长度
    \subsubsection{倒数元素的获取}
      可以用
      \begin{verbatim}
        list[-x]
      \end{verbatim}\par
      来得到列表的倒数第x个元素
    \subsubsection{列表元素的排序与转置}
      使用
      \begin{verbatim}
        list.sort()
      \end{verbatim}\par
      指令使列表数字从小到大排序,同时,
      \begin{verbatim}
        list.reverse()
      \end{verbatim}\par
      可以使列表转置,两者结合,可以写作:
      \begin{verbatim}
        list.sort(reverse = True)
      \end{verbatim}\par
      不难看出,sort操作中reverse指令默认为False
    
\section{元组}
  元组\footnote{注意,一元元组必须表示成tuple = (elementsingle,),否则会被认成数字或字符串}表示为:
  \begin{verbatim}
    tuple = (element_0, element_1, element_2, ……,element_n)
  \end{verbatim}\par
  除了不能修改外,其他诸如查找、切片功能都可以使用\footnote{需要注意的是,列表推导式是坚决不能用来生成元组的,不然为什么叫``列表''推导式呢}\footnote{另外还要注意的是,元组只是不能修改其指向的内存位置,但如果其指向的内存改变了,元组也实际上发生了一定的改变}

\section{获取字符的Unicode编码}
  使用ord(某单个字符)来获得该字符的Unicode编码

\section{序列类型的解包}
  我们可以这样操作\footnote{注意这里iterable是n元的}:
  \begin{verbatim}
    x_0, x_1, ..., x_n-1 = iterable
  \end{verbatim}\par
  来让\verb|x_0到x_n-1|分别对应iterable中的n个元素\footnote{若iterable的元素大于n个,我们可以在\texttt{x\_n-1}左上角加上\texttt{*}号,形如:\texttt{*x\_n-1},来表示后面所有元素组成的一个列表}

\section{保留标识符}
  保留标识符(关键字) 是语言预先定义的具有特殊含义的单词,不能用作变量名、函数名或其他标识符,比如:if, and, for\dots ,可以使用keyword模块的iskeyword函数来判断,操作如下:
  \begin{verbatim}
    import keyword

    keyword.iskeyword(str)
  \end{verbatim}

\section{字符串}
  \subsection{换大小写字母}
    \begin{table}[H]
      \centering
      \caption{字符串大小写转换操作及功能}
      \begin{tabular}{@{}>{\ttfamily}l>{\raggedright\arraybackslash}p{10cm}@{}}
        \toprule
        \textrm{\textbf{操作}} & \textbf{功能描述} \\

        capitalize() & 将字符串的首字母大写,其余字母小写 \\
        casefold()   & 将字符串转换为小写形式(支持特殊字符转换和多语言场景) \\
        title()      & 将字符串中每个单词的首字母大写 \\
        swapcase()   & 将字符串中的大小写互换(大写变小写,小写变大写) \\
        upper()      & 将字符串所有字符转换为大写形式 \\
        lower()      & 将字符串所有字符转换为小写形式 \\
        \bottomrule
      \end{tabular}
    \end{table}

  \subsection{左中右对齐}
    \begin{table}[H]
      \centering
      \caption{字符串对齐与填充操作功能}
      \begin{tabular}{@{}>{\ttfamily}l>{\raggedright\arraybackslash}p{9cm}@{}}
        \toprule
        \textrm{\textbf{操作}} & \textbf{功能描述} \\

        center(width, fillchar=` ') & 将字符串居中,使用指定字符(fillchar的内容,默认是空格,下同)填充两侧至目标宽度 \\
        ljust(width, fillchar=` ')  & 将字符串左对齐,使用指定字符填充右侧至目标宽度 \\
        rjust(width, fillchar=` ')  & 将字符串右对齐,使用指定字符填充左侧至目标宽度 \\
        zfill(width)                & 在字符串左侧填充零至目标宽度(负数符号保留) \\
        \bottomrule
      \end{tabular}
    \end{table}

  \subsection{查找}
    \begin{table}[H]
      \centering
      \caption{python中的查找方法}
      \begin{tabular}{lccc}
        \toprule
        \textbf{方法} & \textbf{功能} & \textbf{未找到行为} & \textbf{方向} \\ 

        \texttt{count()} & 统计出现次数 & 返回 \texttt{0} & 全局 \\
        \texttt{find()} & 从左到右查找某个元素,返回首次找到的索引值 & 返回 \texttt{-1} & 左→右 \\
        \texttt{rfind()} & 从右到左查找某个元素,返回首次找到的索引值 & 返回 \texttt{-1} & 右→左 \\
        \texttt{index()} & 从左到右查找某个元素,返回首次找到的索引值 & 抛出异常 & 左→右 \\
        \texttt{rindex()} & 从右到左查找某个元素,返回首次找到的索引值 & 抛出异常 & 右→左 \\
        \bottomrule
      \end{tabular}
    \end{table}\par
    注意,可以在括号里加入start,end来规定范围(依旧是左闭右开),比如:\verb|count(element, start, end)|

  \subsection{替换}
    \begin{table}[H]
      \centering
      \caption{替换}
      \begin{tabular}{l p{10cm}}
        \toprule
        方法名 & 作用描述 \\

        expandtabs([tabsize=8]) & 将字符串中的制表符(\(\backslash t\))扩展为一定数量的空格,默认制表符大小为8个空格,可通过参数指定。用于文本格式化和处理。 \\
        replace(old, new, count=-1) & 在字符串中替换指定的子字符串,`old`为要被替换的子字符串,`new`为替换的新字符串,`count`指定替换次数,默认-1表示替换所有出现的`old`子字符串。用于修改和清理字符串内容。 \\
        translate(table) & 根据给定的转换表(`table`)对字符串进行字符替换,转换表通常为字典等数据结构,指定字符间的映射关系。常用于批量字符替换或字符编码转换。 \\
        \bottomrule
      \end{tabular}
    \end{table}
    其中,对translate举例:
    \begin{verbatim}
      x = x.translate(str.maketrans(str1, str2, str3))
    \end{verbatim}\par
    这个操作的意义是:先删除x中形如str3的字符,再将其他字符根据str1与str2的对应法则进行更换

  \subsection{判断(返回布尔类型的值)}
    \begin{table}[H]
      \centering
      \caption{Python字符串判断方法汇总}
      \label{tab:string_methods}
      \begin{tabular}{l>{\raggedright\arraybackslash}p{10cm}}
        \toprule
        \textbf{方法} & \textbf{功能说明} \\

        \multicolumn{2}{c}{\textit{前缀/后缀检查}} \\
        \addlinespace
        \texttt{startswith(prefix, start, end)} & 检查字符串是否以指定前缀开头,可选参数限定检测范围 \\
        \texttt{endswith(suffix, start, end)}   & 检查字符串是否以指定后缀结尾,可选参数限定检测范围 \\
        \addlinespace
        \multicolumn{2}{c}{\textit{字符类型判断}} \\
        \addlinespace
        \texttt{isupper()}       & 检测所有字符是否均为大写字母 \\
        \texttt{islower()}       & 检测所有字符是否均为小写字母 \\
        \texttt{istitle()}       & 检测是否符合标题格式(每个单词首字母大写) \\
        \texttt{isalpha()}       & 检测是否仅包含字母字符 \\
        \texttt{isascii()}       & 检测是否均为ASCII字符 \\
        \texttt{isspace()}       & 检测是否均为空白字符(空格/制表符/换行等) \\
        \texttt{isprintable()}   & 检测是否均为可打印字符 \\
        \texttt{isdecimal()}     & 检测是否均为十进制数字(0-9) \\
        \texttt{isdigit()}       & 检测是否均为数字(包括Unicode数字) \\
        \texttt{isnumeric()}     & 检测是否均为数值字符(含分数/罗马数字/中文数字等) \\
        \texttt{isalnum()}       & 检测是否均为字母或数字组合 \\
        \texttt{isidentifier()}  & 检测是否符合变量命名规则(即将此作为变量名是否合法) \\
        \bottomrule
      \end{tabular}
    \end{table}\par
    注意:\par
    1.前两个方法可以在括号内输入元组,这样的话只要有一个成立,布尔类型就为真\par
    2.需要留意\texttt{isalpha()}方法中,空格不是字母字符,因此有空格会返回False的值\par
    3.\texttt{isspace()}方法中,只要能打印且打印出来是空的,均为空白字符,如:制表符,换行符等\par
    4.\texttt{isprintable()}方法中,不可打印字符是指在显示或打印时不会产生可见输出(如文本或符号)的字符,如:换行符、制表符等,而字母、数字、标点符号、空格被认为可打印\par
    5.\texttt{isdigit()}方法当且仅当字符串非空且所有字符都是数字时返回True,其他情况一律返回False,比如:`$2^{2}$'会返回True\par
    6.\texttt{isnumeric()}方法只要是数字就行(英文one, two, three等不算数字)

  \subsection{截取}
    \begin{table}[H]
      \caption{Python字符串处理函数}
      \centering
      \begin{tabular}{ll}
        \toprule
        \textbf{函数} & \textbf{功能} \\

        \texttt{strip(str)} & 移除字符串\textbf{两端}空白/指定字符 \\
        \texttt{lstrip(str)} & 移除字符串\textbf{左端}空白/指定字符 \\
        \texttt{rstrip(str)} & 移除字符串\textbf{右端}空白/指定字符 \\
        \texttt{removeprefix(prefix)} & 移除字符串\textbf{开头}的指定前缀 \\
        \texttt{removesuffix(suffix)} & 移除字符串\textbf{结尾}的指定后缀 \\
        \bottomrule
      \end{tabular}
    \end{table}\par
    需要注意的是,前三个方法是从某个方向开始对字符串的每个元素一一比对是否有str中的字符,若有,就删,然后进入下一个字符,直到没有就停止;而后两个是直接比对字符串前后位置是否与指定前后缀相同,若相同,则执行删除操作,若不符,则不进行操作

  \subsection{拆分}
    \begin{table}[H]
      \centering
      \caption{Python字符串分割方法核心特性对比}
      \begin{tabular}{>{\ttfamily}lccc>{\raggedright\arraybackslash}p{5cm}}
        \toprule
        方法 & 分割次数 & 方向 & 可选参数 & 注意事项 \\

        partition(sep) & 1次 & → & 无 & 
        \begin{itemize}
          \setlength\itemsep{-0.5em}
          \item 未找到分隔符时返回(原字符串, '', '')
          \item 保留分隔符在结果中
          \item 输出三元元组
        \end{itemize} \\
        \addlinespace
        \hline
        rpartition(sep) & 1次 & ← & 无 &  \\
        \addlinespace
        \hline
        split(sep, maxsplit) & 多次 & → & 
        \begin{tabular}{@{}c@{}} 
          sep=None \\ 
          maxsplit=-1 
        \end{tabular} & 
        \begin{itemize}
          \setlength\itemsep{-0.5em}
          \item sep=None时按空白(即空格键)分割(智能去除首尾空白)
          \item maxsplit控制最大分割次数
          \item 不保留分隔符在结果中
          \item 输出表格
        \end{itemize} \\
        \bottomrule
      \end{tabular}
    \end{table}\par
    还有\texttt{splitlines()}的方法,将文字以行划分,输出表格,可选参数为keepend = True/ False,决定是否保留换行符(默认为False)

  \subsection{拼接}
    我们用\texttt{join()}方法将字符串拼接,操作如下:
    \begin{verbatim}
      str_result = str.join(iterable)
    \end{verbatim}\par
    输出结果是将iterable里的元素\footnote{注意,这里iterable里面的元素必须是字符串,如果有其他类型的变量(如:int, float等)必须转化为字符串类型,可选方法有二:一是用列表推导式或者list()函数将iterable转化成列表,二是用\texttt{map(str, iterable)}直接变成str类型}依次排列,并在每两个中间插入字符串str

  \subsection{格式化字符串}
    \subsubsection{初等}
      使用\texttt{format()}方法可以将字符串中花括号内空缺内容填充\footnote{注意,这里面花括号\texttt{{}}仅能选择一个number参数(用以指示使用tuple里的哪个元素)或者一个关键字参数(比如name,然后在tuple里写上name = x,用来专门指代),且不能留有任何空格;另外,若两种参数同时出现,那么位置参数必须在关键字参数之前,否则就会出错},形式如下:
      \begin{verbatim}
        str(中间含{}).format(tuple)
      \end{verbatim}\par
      这里必须满足\{\}数不得多于tuple的元素数,否则会报错;另外,要想输出\{str1\}可以用关键字参数+`\{str1\}'或者直接在str中使用\{\{str1\}\}
    \subsubsection{进阶}
      \texttt{format()}方法有很多可选参数,可以表示成如下的``式子'':
      \begin{verbatim}
        formatted_string = `{:{fill}{align}{sign}(#)(0){width}{grouping_option}.{precision}{type}}'.format(str, fill = ..., align = ..., )
      \end{verbatim}\par
      其中,可选字符的效果如下:
      \begin{table}[H]
        \centering
        \caption{format 方法可选参数详解}
        \begin{tabular}{>{\ttfamily}l >{\raggedright\arraybackslash}p{8cm}}
          \toprule
          \textbf{参数} & \textbf{描述} \\
          
          fill     & 指定填充字符(默认空格),如 \texttt{:a<5} → "hello" → "helloa" \\
          align    & 对齐方式:\texttt{<}(左)、\texttt{>}(右)、\texttt{\^}(中) \\
          sign     & 符号显示:\texttt{+}(总显示)、\texttt{-}(仅负)、空格(正数留空) \\
          \#       & 添加进制前缀(0x/0o/0b),如 \texttt{\{:x\}} → 0xff \\
          0        & 用0替代空格填充,如 \texttt{:05d} → 42 → "00042" \\
          width    & 最小输出宽度,如 \texttt{:10} → "hello" → "     hello" \\
          grouping\_option & 数字分组(如千位分隔,可选,和\_),\texttt{:,.2f} → 1234.56 → "1,234.56" \\
          precision & 浮点数精度,如 \texttt{:.3f} → 3.14159 → "3.142" \\
          type     & 数据类型:\texttt{s}(字符串)、\texttt{d}(整)、\texttt{f}(浮点)、\texttt{b/o/x}(进制) \\
          \bottomrule
        \end{tabular}
      \end{table}
      注意:\par
      1.fill的内容必须为单位长度的字符串或0123456789中的一个数字\par
      2.为align, sign赋值的内容必须是字符串\par
      当然,也可以使用f-string的方法来简化代码:
      \begin{verbatim}
        formatted_string = f.`{str:{fill}{align}{sign}(#)(0){width}{grouping_option}{precision}{type}.{precision}{type}}'
      \end{verbatim}\par
      注意这也要有一个\{\}

\section{序列}
  \subsection{运算符}
    \subsubsection{is运算符}
      is运算符用以指向两个变量是否指向同一个内存位置\footnote{想调用内存位置可以用函数id(target)}(is not与之相反)\par
      同一个字符串只在一个内存中储存,一样的一张列表会储存在不同的内存位置\par
      而对于直接将某个列表乘以n的情况,实际上是复制其内存位置;而用列表推导式则不会出现这个问题
    \subsubsection{in运算符}
      判断某个元素是否包含于另一个序列
    \subsubsection{del运算符}
      del运算符可以用来执行删除操作,可以直接删除变量,如:\verb|del x|就直接删除了变量x,也可以用于删除列表中的元素,写法和切片类似(效果就是把切片赋值为空)

  \subsection{函数}  
    \subsubsection{min(),max()}
      字如其名,可以直接输入参数,或者比较的元素大小,比如:\texttt{min(iterable)}注意字符串是比较其编码值\par
      当然,有时无法返回结果,我们可以写\texttt{min(iterable, default = str)},这样,无法输出对象时,就会输出default里面的值
    \subsubsection{sum(iterable, start = number)}
      start意为从number开始加
    \subsubsection{sorted(iterable, key = function, reverse = Ture/False)}
      注意,list.sort()会直接改变list,而sorted(list)会形成一个新的列表;sort()只能处理list,sorted()可以处理一切,并传出list
    \subsubsection{reversed(iterable)}
      注意这个会传出一个迭代器,要想查看需将这个转化成list, tuple, string等可视化对象
    \subsubsection{all(),any()}
      用来判断里面的元素是否全为真/是否存在真
    \subsubsection{enumerate()}
      操作:
      \begin{verbatim}
        enumerate(iterable, start = number)
      \end{verbatim}\par
      结果(迭代器\footnote{迭代器只能使用一次,用过即消失,想让可迭代对象变成迭代器,可以使用iter(iterable)函数,将其转化为迭代器},需要转换成list, tuple, string的变量来查看):
      \begin{verbatim}
        (number, element_1), (number + 1, element_2), ...(以此类推,这些都是二元元组)
      \end{verbatim}
    \subsubsection{zip()}
      zip()函数将几个可迭代对象的第i个元素打包成第i个元组传出(迭代器,需要转换成list, tuple, string的变量来查看),比如:
      \begin{verbatim}
        x = [1, 2, 3, 4 ,5]
        y = [1, 1, 4, 5, 1]
        z = [3, 7, 8, 1, 0]

        zipped = list(zip(x, y, z))
      \end{verbatim}\par
      会输出:
      \begin{verbatim}
        [(1, 1, 3), (2, 1, 7), (3, 4, 8), (4, 5, 1), (5, 1, 0)]
      \end{verbatim}\par
      注意,如果几个iterable的元素数目不同,则在最短序列时停止并输出结果,要是想用那些被忽略的值,可以使用itertools模块的\texttt{zip\_longest()}函数,操作如下:
      \begin{verbatim}
        import itertools

        itertools.zip_longest(x, y, z, ..., fillvalue = number/str)
      \end{verbatim}\par
      会将空缺处以fillvalue填充,默认值是bool类型的None
    \subsubsection{map(function, iterable)}
      对所有iterable的元素依次进行function操作,然后形成一个可迭代对象(迭代器,需要转换成list, tuple, string的变量来查看)
    \subsubsection{filter(function, iterable)}
      与map()类似,但只返回结果为真的变量,其他删了不要
    \subsubsection{next(iterator, str)}
      将迭代器中的元素一一拿出来,注意,每次只返回一个变量,下次返回下一个,不输入str时,没有变量时会出现异常,若输入可控变量str,则会返回str

  \subsection{将可迭代对象转化为某几种序列的函数}
    可以使用\texttt{list(), tuple(), str()}函数将可迭代对象分别转化为列表,元组和字符串\par
    注意:前两个互换可以理解为改变括号,前两个变成str是直接在两边加引号,而str变成前两个是将每个字符取出来作为list或者tuple里的元素

\section{字典}
  \subsection{概述}
    字典是一种键值对存储结构,形如:
    \begin{verbatim}
      dictionary = {key1: value1, key2: value2, ...}
    \end{verbatim}\par
    需要注意,key的值互不相同;可以通过key的值获得相应的value:
    \begin{verbatim}
      dictionary[keyN] = valueN
    \end{verbatim}\par
    这个操作同样可用于修改key所对应的value值,若keyN不存在,则会将该键值对写入

  \subsection{字典的构建}
      \subsubsection{直接构建}
        \begin{verbatim}
          dictionary = {key1: value1, key2: value2, ...}
        \end{verbatim}\par
        这里value的数据类型没有限制,而key的类型是不可变类型(字符串、数字、元组等)
      \subsubsection{利用\texttt{dict()}函数来构建}
        第一种方法和上面一样;\par
        第二种(有点像赋值)是:
        \begin{verbatim}
          dictionary = dict(key1 = value1, key2 = value2, ...)
        \end{verbatim}\par
        特别注意key的内容必须为能在加引号后变为字符串的一串字符且不能为数字(函数知识);第三种(类似于传入元组)是:
        \begin{verbatim}
          dictionary = dict((key1, value1), (key2, value2), ...)
        \end{verbatim}\par
        然后是先用\texttt{zip()}函数构建成打包的元组再传入:
        \begin{verbatim}
          dictionary = dict(zip(iterable1, iterable2))
        \end{verbatim}\par

  \subsection{操作}
    \subsubsection{del(), clear(), len(), in, list(), iter()}
      与之前相同;注意,\texttt{list(), iter()}只会对key进行操作
    \subsubsection{几个方法}
      \begin{table}[H]
        \centering
        \caption{Python 字典核心方法}
        \begin{tabular}{lp{12cm}}
          \toprule
          \textbf{方法} & \textbf{功能} \\
          
          \texttt{fromkeys(keys, value)} & 创建新字典,所有key的值对应为value\\
          \texttt{pop(key, default)} & 删除指定键,若没有,返回default\\
          \texttt{popitem()} & 删除最后插入的键值对\\
          \texttt{update(dictionary)} & 批量更新字典\\
          \texttt{get(key, default)} & 查找key对应的value值,没有就传出default\\
          \texttt{setdefault(key, default)} & 若字典中有key,不响应;若无key,形成(key, default)的键值对\\
          \texttt{keys()} & 返回字典所有键的动态视图对象,实时反映字典变化 \\
          \texttt{values()} & 返回字典所有值的动态视图对象,实时反映字典变化 \\
          \texttt{items()} & 返回字典所有键值对(元组形式)的动态视图对象,实时反映字典变化 \\
          \bottomrule
        \end{tabular}
      \end{table}
  
  \subsection{字典的嵌套}
    将value赋值成一个字典即可,查看操作与列表相同

  \subsection{字典推导式}
    \begin{verbatim}
      dictionary = {key: value for element in iterable if condition}
    \end{verbatim}\par
    或者:
    \begin{verbatim}
      dictionary_ = {key_: value_ for key, value in dictionary.items() if condition}
    \end{verbatim}
    
\section{集合} 
  \subsection{集合的构建}
    \begin{verbatim}
      set = {element1, element2, ...}
    \end{verbatim}\par
    或者用\texttt{set()}函数:
    \begin{verbatim}
      set = set(iterable)
    \end{verbatim}\par
    这里会构建一个iterable所有元素的集合(会自动去除重复的元素)。注意,集合有无序性,因此不能用下标索引。\par
    这里的集合是可变对象,要想构建一个不可变集合,可以使用frozenset()方法,用法与set()相同。\par
    注意,集合也有推导式

  \subsection{判断}
    \begin{table}[H]
      \centering
      \caption{Python集合核心方法}
      \begin{tabular}{lp{8cm}}
        \toprule
        \textbf{方法} & \textbf{核心功能描述} \\
        \midrule
        \texttt{isdisjoint(other)} & 检查两集合是否无交集(无共同元素) \\
        \texttt{issubset(other)} & 检查当前集合是否为另一集合的子集 \\
        \texttt{issuperset(other)} & 检查当前集合是否为另一集合的超集 \\
        \texttt{union(*others)} & 返回当前集合与其他集合的并集 \\
        \texttt{intersection(*others)} & 返回当前集合与其他集合的交集 \\
        \texttt{difference(*others)} & 返回当前集合与其他集合的差集 \\
        \texttt{symmetric\_difference(other)} & 返回当前集合与其他集合的对称差集 \\
        \bottomrule
      \end{tabular}
    \end{table}
  其它快捷方法:
    \begin{table}[H]
      \centering
      \caption{快捷方案}
      \begin{tabular}{lp{4cm}}
        \toprule
        \textbf{方法} & \textbf{核心功能描述} \\
        \midrule
        <, >, = & 检查两集合包含关系 \\
        setA | setB & 取并集 \\
        setA \& setB & 取交集 \\
        setA - setB & 取差集 \\
        setA \^ setB & 取对称差集 \\
        \bottomrule
      \end{tabular}
    \end{table}

  \subsection{其他操作}
    \subsubsection{update()操作}
      有intersection\_update(), difference\_update(), symmetric\_difference\_update()方法,相当于将原来的集合赋值给判断后输出的集合
    \subsubsection{add(element), remove(element), discard(element), pop()}
      set中没有element时,remove()会报错,而discard()不会;pop()会随机删一个元素
    
\section{函数}
  \subsection{写法}
    \begin{verbatim}
      def function(parameter1, parameter2, ...):
        content
    \end{verbatim}\par
    调用时就写:
    \begin{verbatim}
      function(parameter1, parameter2, ...)
    \end{verbatim}\par
    注意,以上parameterN可以写作parameterN = ...,这也叫做关键字参数。写函数时,关键字参数必须要出现在位置参数后面(可以用*来隔断);当然,也可以通过这个给函数提供默认值

  \subsection{不确定元素个数的输入方式}
    \begin{verbatim}
      def function(*tuple_name, parameter1, parameter2, ...):
        content
    \end{verbatim}\par
    注意,这里tuple\_name的意思是传入的参数以元组的形式储存,输入时仍正常输入即可;后面的参数必须用关键字参数

  \subsection{函数中字典的构建}
    \begin{verbatim}
      def function(**keywords):
        content
    \end{verbatim}\par
    注意,这里要想调用,必须写作:
    \begin{verbatim}
      function(key1 = value1, key2 = value2, ...)
    \end{verbatim}\par
    这里key参数若为字符串,不能加引号

  \subsection{return}
    作用:\par
    1.返回结果:将函数内部计算或处理的结果传递给外部调用者\par
    2.终止函数执行:一旦执行return,函数会立即结束运行,后续代码不再执行;若无return,代码自动返回None

  \subsection{作用域}
    在函数内部定义的变量仅能在该函数内部作用,不能在函数外部出现;而在函数外部出现的变量没有限制\par
    需要注意的是,如果一个变量同时在函数内外出现,则内外互不影响(若内部未进行赋值等操作,内部就调用外部的值;若进行操作,就使用内部操作的结果,函数运行完毕后,变量的值变回原来的值)\par
    但是如果想在函数里面定义或者改变一个全局变量,可以使用global语句:
    \begin{verbatim}
      def function():
        global variation
        content
    \end{verbatim}\par
    但不提倡这样做

  \subsection{嵌套函数}
  \begin{verbatim}
    def functionA():
      content
      def functionB():
        content
      content
  \end{verbatim}\par
  注意,不能在外部直接调用functionB,要想使用functionB必须将调用语句写在functionA里面,嵌套函数中,变量升一级用nonlocal语句

  \subsection{LEGB法则}
    作用‘‘效果’’:local > enclose > global > build-in,意思是:当前后可能冲突时,会优先选择``更大的''来进行作用

  \subsection{闭包}
    通过闭包操作,可以将一个变量单独的储存在某个包内,避免外部参数对其的影响,一般写法为:
    \begin{verbatim}
      def functionA(parameters):
        content(包含定义局域参数parameter_x的语句)

        def functionB(parameters):
          content(包含:nonlocal parameter_x语句)
          return ...

        content
        return functionB

      fun = functionA(parameters)

      fun(parameters)
    \end{verbatim}\par
    这里,我们每次调用fun时,相当于是在使用functionB进行操作,而内置于functionB(引用functionA的parameter\_x)的参数没有被删除,可以经过上一个值来进行重复使用;如果将functionA多次赋值给不同的参数,里面的parameter\_x互不影响

\section{*的用法}
  \subsection{解包}
    对list, tuple使用一个*号,对字典使用2个*号并且只会将key参数解包
  
  \subsection{表示不确定元素个数的输入方式}
    见函数章节




\end{document}
